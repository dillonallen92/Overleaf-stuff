\documentclass{article}
\usepackage[utf8]{inputenc}
\usepackage{amsmath}

\title{Calculus 2 Integrating Factor}
\author{Dillon Allen }
\date{October 2018}

\begin{document}

\maketitle

\section{Integrating Factor and Examples}
First we will start off with the separable differential equation

$$ \frac{dy}{dx} = \frac{5 cos\left(x\right)}{2y}$$

Separate variables to get 

$$ 2y dy = 5 cos\left(x \right) dx$$

Integrating both sides we have:

$$ y^2 = 5 sin  \left( x \right) + C $$

Therefore

$$ y = \pm \sqrt{ 5 sin \left( x \right) + C } $$

That is the general solution to the separable differential equation shown above. But what happens if we encounter a differential equation of the form 
$$ y' + 5xy = 2 x^2$$

This differential equation isn't separable, so we need a new method to solving these. This method is called \textbf{integrating factor}. 

The differential equation above is known as a first order linear differential equation. We have a 5 step process process for this.

1) turn the differential equation into standard form

standard form is of the form

$$ y' + p\left(x\right)y = q\left(x\right) $$

2) define your $p\left(x\right)$ and $q\left(x\right)$

3) find the integrating factor, which is defined as

$$ \mu\left(x\right) = \exp\left[\int p\left(x\right)dx\right]$$


4) multiply $\mu$ through the equation in (1) and decompose the left hand side down to a product rule

$$ \frac{d}{dx}\left[\mu\left(x\right)y\right] = \mu\left(x\right)q\left(x\right) $$

5) The general solution of the differential equation is

$$ y = \frac{1}{\mu\left(x\right)}\int \left[ \mu\left(x\right)q\left(x\right) + C\right]$$

\textbf{Example}:

Solve the differential equation 
$$ xy' + y = x^2 + 1$$

1) Putting into standard form we have 

$$y' + \frac{1}{x}y = x + \frac{1}{x} $$

2) Identify your $p\left(x\right)$ and $q\left(x\right)$

$$ p\left(x\right) = \frac{1}{x} $$

$$ q\left(x\right) = x + \frac{1}{x} $$

3) find $\mu \left(x \right)$

$$ \mu\left(x\right) = \exp\left[ \int \frac{1}{x} \mathrm{d}x \right] $$

$$ \mu \left( x \right) = x $$

4) Multiply $\mu \left( x \right)$ through and decompose

$$ x y' + y = x^2 + 1 $$

$$ \mathrm{d}\left[ x y \right] = \left( x^2 + 1 \right) \mathrm{d}x $$

5) Integrating both sides, we get 

$$ y = \frac{1}{x} \int \left( x^2 + 1 \right) \mathrm{d}x $$

\begin{equation}
\boxed{y = \frac{1}{3} x^2 + 1 + \frac{C}{x} }
\end{equation}

\newpage

\textbf{Example}

Solve the equation

$$ y' + \frac{3}{x} y = \frac{e^x}{x^3} $$

1) The equation is already in standard form so we do not need to do anything for step 1

2) Identify $p \left( x \right)$ and $q \left( x \right)$

$$ p \left( x \right) = \frac{3}{x}$$

$$ q\left( x \right) = \frac{e^x}{x^3} $$ 


3) Find $\mu \left( x \right)$

$$ \mu \left( x \right) = \exp\left[ \int \frac{3}{x} \mathrm{d}x \right]$$

$$ \mu \left( x \right) = x^3 $$ 

4) Multiply through and decompose

$$ x^3 y' + 3x^2y = e^x $$

$$ \mathrm{d}\left[ x^3 y \right] = e^x \mathrm{d}x $$

5) Integrating both sides and solving for $y$, we get

$$ x^3y = e^x + C $$
\begin{equation}
 \boxed{y = \frac{e^x + C}{x^3}}
\end{equation}
\textbf{Example}

Sovle the following differential equation

$$ \left( x + 1 \right) y' - 3 y = \left( x + 1 \right)^5 $$ 

1) Putting into standard form, we have 

$$ y' - \frac{3}{x+1} = \left( x + 1 \right)^4 $$ 

2) Identify $ p\left(x\right) $ and $ q\left(x\right) $

$$ p\left(x\right) = \frac{-3}{x+1} $$

$$ q\left(x \right) = \left(x+1\right)^4 $$

3) Find $ \mu\left( x \right) $

$$ \mu \left( x \right) = \exp\left[ \int \frac{-3}{x+1} \mathrm{d}x \right] $$

$$ \mu \left( x \right) = \left( x + 1 \right) ^{-3} $$

4) Multiply through and decompose

$$ \left( x + 1 \right)^{-3}y' - 3 \left( x + 1 \right)^{-4} = x + 1 $$

$$ \mathrm{d}\left[ \left( x + 1 \right)^{-3} y \right] = \left( x + 1 \right) \mathrm{d}x $$

5) Integrate both sides and solving for y, we get 

\begin{equation}
    \boxed{y = \left( x + 1 \right)^{3}\left( \frac{1}{2}x^2 + x + C \right)}
\end{equation}

\section{Derivation}

\end{document}
